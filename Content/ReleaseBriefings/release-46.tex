\input{KokkosTutorial_PreTitle}
\usepackage{tikz}
\graphicspath{{4_6/figures/}}

%disclaimer for Sandia. uncomment and the whole blob goes away @ b80c116300122
\def\sandid{SANDXXXX PE}

% \title{Performance Portability with Kokkos}
\title{Kokkos 4.6 Release Briefing}

%BAD misuse of author field
\author{New Capabilities}

\date{04/10/2025}

\input{KokkosTutorial_PostTitle}

\shorttrue
\mediumfalse
\fullfalse

\begin{document}

\begin{frame}
  \titlepage
\end{frame}


\begin{frame}[fragile]{Outline}

  \textbf{4.6 Release Highlights}

  \begin{itemize}
    \item{Organizational}
    \item{Feature Highlights}
    \item{General Enhancements}
    \item{Backend updates}
    \item{\hyperlink{sec:dualview}{\texttt{DualView} changes}}
    \item{Build system updates}
    \item{Deprecations and other breaking changes}
    \item{Bug Fixes}
  \end{itemize}

\end{frame}

\begin{frame}{Find More}

  \textbf{Online Resources}:

  \begin{itemize}
    \item \url{https://github.com/kokkos}:
          \begin{itemize}
            \item Primary Kokkos GitHub Organization
          \end{itemize}
    \item \url{https://github.com/kokkos/kokkos-tutorials/wiki/Kokkos-Lecture-Series}:
          \begin{itemize}
            \item{Slides, recording and Q\&A for the Full Lectures}
          \end{itemize}
    \item \url{https://kokkos.org/kokkos-core-wiki}:
          \begin{itemize}
            \item Wiki including API reference
          \end{itemize}
    \item \url{https://kokkosteam.slack.com}:
          \begin{itemize}
            \item Slack workspace for Kokkos.
            \item Please join: fastest way to get your questions answered.
            \item Can whitelist domains, or invite individual people.
          \end{itemize}
  \end{itemize}

\end{frame}

\begin{frame}[fragile]{Kokkos Usage}
  \textbf{Would like to strengthen community bonds and discoverability}

  \vspace{10pt}
  \textit{List of Applications and Libraries}
  \begin{itemize}
    \item Add your app to \url{https://github.com/kokkos/kokkos/issues/1950}
    \item We are planning to add that to a Kokkos website.
    \item Helps people discover each other when working on similar things.
  \end{itemize}

  \vspace{10pt}
  \textit{GitHub Topics}
  \begin{itemize}
    \item Use \textit{kokkos} tag on your repos.
    \item If you click on the topic you get a list of all projects on github with that topic.
  \end{itemize}
\end{frame}


%==========================================================================

\begin{frame}[fragile]

  {\Huge Organizational}

  \vspace{10pt}

  \textbf{Content:}
  \begin{itemize}
    \item HPSF and Kokkos Meeting 2025
    \item Targeting C++20 for Kokkos 5.0
    \item Makefile deprecation
  \end{itemize}

\end{frame}

%==========================================================================

\begin{frame}[fragile]{Kokkos User Group Meeting 2025}
\begin{center}
\textbf{Kokkos User Group Meeting 2025 @ HPSF Conference}
\end{center}

\begin{itemize}
\item{\textit{When:} May 5th-8th 2025}
\item{\textit{Where:} Chicago}
\item{\textit{What:} 2-days HPSF plenary + 2-days Project meetings}
\item{\textit{KUG-Content:} Focused on user experiences
\begin{itemize}
   \item{How do you leverage Kokkos?}
   \item{What are pain points?}
   \item{Kokkos based libraries of interest for the community}
\end{itemize}
}
\end{itemize}


\vspace{10pt}

\begin{center}
\textit{Registration open now!}
\end{center}
\end{frame}


\begin{frame}[fragile]{Kokkos 5 and ISO C++20}
\begin{center}
\textbf{Kokkos 5 is comming Summer 2025}

\vspace{0.5cm}
\textbf{We will require C++20!}
\end{center}

\textit{Start preparing now:}
\begin{itemize}
  \item{Check availability of compilers on your systems}
  \item{Test with C++20 enabled: start with a CPU build}
  \item{Minimum Compiler requirements will change (more details later)}
\end{itemize}

\vspace{0.5cm}
\begin{center}
\textit{Nothing wrong for your project to require C++20 now if you feel ready!}
\end{center}
\end{frame}

\begin{frame}[fragile]{Makefile deprecation}
\begin{center}
\textbf{Makefile is officially scheduled for deprecation in the next major release}

\textit{Start preparing now:}
\begin{itemize}
  \item{Check if you can transition to CMake}
  \item{Comment on pinned issue 7610}
\end{itemize}
\end{center}

\end{frame}

\begin{frame}[fragile]{Open SSF Scorecard}
\begin{center}
\textbf{We reached ``passed'' on the OSSF Best Practices Program}
\href{https://www.bestpractices.dev/en/projects/9344}{www.bestpractices.dev}

\vspace{0.5cm}
\textit{This means Kokkos is continuously tracking and openly reporting the conformity with open source software practices.}
\end{center}

\end{frame}

%==========================================================================

\begin{frame}[fragile]

  {\Huge Feature highlights}

  \vspace{10pt}

\end{frame}

%==========================================================================

% Examples

% note: always keep the [fragile] for your frames!

%\begin{frame}[fragile]{Example list}
%  \begin{itemize}
%      \item Item 1
%      \item Item 2 with some \texttt{code}
%      \begin{itemize}
%        \item Sub-item 2.1
%        \item Sub-item 2.2
%      \end{itemize}
%  \end{itemize}
%\end{frame}

%\begin{frame}[fragile]{Example code}
%    \begin{code}[keywords={std}]
%        #include <iostream>
%        
%        int main() {
%            std::cout << "hello world\n";
%        }
%    \end{code}
%\end{frame}

%\begin{frame}[fragile]{Example table}
%    \begin{center}
%        \begin{tabular}{l|l}
%            a & b \\\hline
%            c & d
%        \end{tabular}
%    \end{center}
%\end{frame}

%==========================================================================

\begin{frame}[fragile]\label{sec:new_features}

  {\Huge Kokkos::Graph features}

  \vspace{10pt}

\end{frame}

\begin{frame}[fragile]{Kokkos::Graph recap}
 \begin{itemize}
     \item describes asynchronous workloads organised as a direct acyclic graph (DAG)
     \item executed using \texttt{submit()}, possibly many times, observing dependencies
      \begin{code}[keywords={auto}]
    ...
    auto graph = Kokkos::Experimental::create_graph(exec, [&](auto root) {
      auto node_A = root.then_parallel_for(
                    policy_t(exec, 0, 1),
                    FetchValuesAndContribute(data, index_A, value_A));
      auto node_B = root.then_parallel_for(
                    policy_t(exec, 0, 1),
                    FetchValuesAndContribute(data, index_B, value_B));}
    ...
    graph.submit(exec);
     \end{code}
 \end{itemize}
\end{frame}

\begin{frame}[fragile]{Kokkos::Graph new features}
 \begin{itemize}
   \item \texttt{then}-node: adds callable to be executed once
   \item Executed in the \texttt{ExecutionSpace} the graph is submitted to
     \begin{code}[keywords={auto}]
    ...
    auto graph = Kokkos::Experimental::create_graph(exec, [&](auto root) {
      auto node_A = root.then_parallel_for(
                    policy_t(exec, 0, 1),
                    FetchValuesAndContribute(data, index_A, value_A));
      auto node_B = node_A.then(
                    FetchValuesAndContribute(data, index_B, value_B));}
    ...
    graph.submit(exec);
     \end{code}
 \end{itemize}
\end{frame}

\begin{frame}[fragile]{Kokkos::Graph new features}
 \begin{itemize}
   \item Interoperabiliy: create a Kokkos::Graph from CUDA/HIP/SYCL graph
   \item Does not yet allow linking Kokkos nodes to native nodes
     \begin{code}[keywords={auto}]
    ...
    cudaGraph_t native_graph = nullptr;
    KOKKOS_IMPL_CUDA_SAFE_CALL(cudaGraphCreate(&native_graph, 0));
    auto graph_from_native =
      Kokkos::Experimental::create_graph_from_native(exec, native_graph);
    auto root = Kokkos::Impl::GraphAccess::create_root_ref(graph_from_native);
    root.then_parallel_for(1, Increment<view_t>{data});
    ...
    graph_from_native.submit(exec);
     \end{code}
 \end{itemize}
\end{frame}

%==========================================================================

%==========================================================================

\begin{frame}[fragile]

  {\Huge General Enhancements}

  \vspace{10pt}

\end{frame}

\begin{frame}[fragile]{\texttt{inclusive\_scan} performance improvements (3/7)}
  With the Cuda and HIP backends, \texttt{Kokkos::Experimental::inclusive\_scan} now calls the vendor versions from Thrust
  \begin{itemize}
     \item The vendor versions are faster than the \texttt{Kokkos::parallel\_scan} based default implementation
     \item Both backends call the vendor versions by default
       \begin{itemize}
         \item The HIP version is called when \texttt{Kokkos\_ENABLE\_ROCTHRUST} is \texttt{ON} (which is the default)
       \end{itemize}
  \end{itemize}
\end{frame}

\begin{frame}[fragile]{Reduce tooling interface overhead (4/7)}
  Reduced the overhead of Kokkos tools related checks
  \begin{itemize}
      \item Store the information whether Kokkos tools are enabled after each modification to the tools callbacks
      \item Previously, this value was recomputed everytime an event of interest (\texttt{parallel\_for}, \texttt{fence}, etc.) occured
      \item Mostly noticeable in small serial kernels (around 100 elements)
  \end{itemize}
\end{frame}
%==========================================================================

% Examples

% note: always keep the [fragile] for your frames!

%\begin{frame}[fragile]{Example list}
%  \begin{itemize}
%      \item Item 1
%      \item Item 2 with some \texttt{code}
%      \begin{itemize}
%        \item Sub-item 2.1
%        \item Sub-item 2.2
%      \end{itemize}
%  \end{itemize}
%\end{frame}

%\begin{frame}[fragile]{Example code}
%    \begin{code}[keywords={std}]
%        #include <iostream>
%        
%        int main() {
%            std::cout << "hello world\n";
%        }
%    \end{code}
%\end{frame}

%\begin{frame}[fragile]{Example table}
%    \begin{center}
%        \begin{tabular}{l|l}
%            a & b \\\hline
%            c & d
%        \end{tabular}
%    \end{center}
%\end{frame}

%==========================================================================

\begin{frame}[fragile]{Fix a warning from kokkos\_check}
 \begin{itemize}
    \item \texttt{kokkos\_check}: makes sure that Kokkos was built with the requested backends and target architectures and generates a fatal error if it was not.
    \item Fix a warning when a user calls the cmake function \texttt{kokkos\_check} from a \texttt{<PackageName>Config.cmake} file
    {\tiny \begin{verbatim}
CMake Warning (dev) at /usr/share/cmake-3.22/Modules/FindPackageHandleStandardArgs.cmake:438 (message):
  The package name passed to `find_package_handle_standard_args`
  (Kokkos_DEVICES) does not match the name of the calling package (SomePackage).
  This can lead to problems in calling code that expects `find_package`
  result variables (e.g., `_FOUND`) to follow a certain pattern.
Call Stack (most recent call first):
  ... /kokkos/lib/cmake/Kokkos/KokkosConfigCommon.cmake:110 (find_package_handle_standard_args)
  ...
This warning is for project developers.  Use -Wno-dev to suppress it.
    \end{verbatim}}
 \end{itemize}
\end{frame}


%==========================================================================

%==========================================================================

\begin{frame}[fragile]

  {\Huge Backend Updates}

  \vspace{10pt}

  \textbf{Content:}
  \begin{itemize}
    \item Backend Updates CUDA
    \item Backend Updates HIP
    \item Backend Updates SYCL
  \end{itemize}

\end{frame}

%==========================================================================

% Examples

% note: always keep the [fragile] for your frames!

%\begin{frame}[fragile]{Example list}
%  \begin{itemize}
%      \item Item 1
%      \item Item 2 with some \texttt{code}
%      \begin{itemize}
%        \item Sub-item 2.1
%        \item Sub-item 2.2
%      \end{itemize}
%  \end{itemize}
%\end{frame}

%\begin{frame}[fragile]{Example code}
%    \begin{code}[keywords={std}]
%        #include <iostream>
%        
%        int main() {
%            std::cout << "hello world\n";
%        }
%    \end{code}
%\end{frame}

%\begin{frame}[fragile]{Example table}
%    \begin{center}
%        \begin{tabular}{l|l}
%            a & b \\\hline
%            c & d
%        \end{tabular}
%    \end{center}
%\end{frame}

%==========================================================================


%==========================================================================


%==========================================================================

\begin{frame}[fragile]\label{sec:dualview}

  {\Huge DualView changes}

  \vspace{10pt}

\end{frame}


%==========================================================================

% Examples

% note: always keep the [fragile] for your frames!

\begin{frame}[fragile]{DualView changes}
  Deprecate direct access to \texttt{d\_view} and \texttt{h\_view}
  \begin{itemize}
    \item Modifying the allocations in d\_view and h\_view is dangerous, especially if \texttt{modify} and \texttt{sync} are skipped
    \item Use \texttt{view\_host} and \texttt{view\_device} instead
    \item These two functions return by value with deprecated code enabled and by reference otherwise. This might have perfomance implications if used extensively, e.g., in loop bounds.
  \end{itemize}
\end{frame}

%==========================================================================


%==========================================================================

%==========================================================================

\begin{frame}[fragile]

  {\Huge Build Systems Updates}

  \vspace{10pt}

\end{frame}

%==========================================================================

% Examples

% note: always keep the [fragile] for your frames!

\begin{frame}[fragile]{New build system features}
  \begin{block}{Architecture}
  \begin{itemize}
      \item Add support for Zen 4 AMD microarchitecture \hyperlink{https://github.com/kokkos/kokkos/pull/7550}{\#7550}
      \item Enable NVIDIA Grace architecture with NVHPC \hyperlink{https://github.com/kokkos/kokkos/pull/7858}{\#7858}
      \item Support static library builds when using CUDA as CMake language [\#7830](https://github.com/kokkos/kokkos/pull/7830)
  \end{itemize}
  \end{block}

  \begin{block}{CUDA}
  \begin{itemize}
      \item It is possible to build static library builds when enabling CUDA as CMake language \hyperlink{https://github.com/kokkos/kokkos/pull/7830}{\#7830}
  \end{itemize}
  \end{block}
\end{frame}

%\begin{frame}[fragile]{Example code}
%    \begin{code}[keywords={std}]
%        #include <iostream>
%        
%        int main() {
%            std::cout << "hello world\n";
%        }
%    \end{code}
%\end{frame}

%\begin{frame}[fragile]{Example table}
%    \begin{center}
%        \begin{tabular}{l|l}
%            a & b \\\hline
%            c & d
%        \end{tabular}
%    \end{center}
%\end{frame}

%==========================================================================


%==========================================================================

%==========================================================================

\begin{frame}[fragile]

  {\Huge Deprecations and other breaking changes}

  \vspace{10pt}

\end{frame}

%==========================================================================

% Examples

% note: always keep the [fragile] for your frames!

%\begin{frame}[fragile]{Example list}
%  \begin{itemize}
%      \item Item 1
%      \item Item 2 with some \texttt{code}
%      \begin{itemize}
%        \item Sub-item 2.1
%        \item Sub-item 2.2
%      \end{itemize}
%  \end{itemize}
%\end{frame}

%\begin{frame}[fragile]{Example code}
%    \begin{code}[keywords={std}]
%        #include <iostream>
%        
%        int main() {
%            std::cout << "hello world\n";
%        }
%    \end{code}
%\end{frame}

%\begin{frame}[fragile]{Example table}
%    \begin{center}
%        \begin{tabular}{l|l}
%            a & b \\\hline
%            c & d
%        \end{tabular}
%    \end{center}
%\end{frame}

%==========================================================================


%==========================================================================

%==========================================================================

\begin{frame}[fragile]

  {\Huge Deprecations}

  \vspace{10pt}

\end{frame}


%==========================================================================

% Examples

% note: always keep the [fragile] for your frames!

%\begin{frame}[fragile]{Example list}
%  \begin{itemize}
%      \item Item 1
%      \item Item 2 with some \texttt{code}
%      \begin{itemize}
%        \item Sub-item 2.1
%        \item Sub-item 2.2
%      \end{itemize}
%  \end{itemize}
%\end{frame}

%\begin{frame}[fragile]{Example code}
%    \begin{code}[keywords={std}]
%        #include <iostream>
%        
%        int main() {
%            std::cout << "hello world\n";
%        }
%    \end{code}
%\end{frame}

%\begin{frame}[fragile]{Example table}
%    \begin{center}
%        \begin{tabular}{l|l}
%            a & b \\\hline
%            c & d
%        \end{tabular}
%    \end{center}
%\end{frame}

%==========================================================================


%==========================================================================

%==========================================================================

\begin{frame}[fragile]

  {\Huge Bug Fixes}

  \vspace{10pt}

\end{frame}

%==========================================================================

% Examples

% note: always keep the [fragile] for your frames!

%\begin{frame}[fragile]{Example list}
%  \begin{itemize}
%      \item Item 1
%      \item Item 2 with some \texttt{code}
%      \begin{itemize}
%        \item Sub-item 2.1
%        \item Sub-item 2.2
%      \end{itemize}
%  \end{itemize}
%\end{frame}

%\begin{frame}[fragile]{Example code}
%    \begin{code}[keywords={std}]
%        #include <iostream>
%        
%        int main() {
%            std::cout << "hello world\n";
%        }
%    \end{code}
%\end{frame}

%\begin{frame}[fragile]{Example table}
%    \begin{center}
%        \begin{tabular}{l|l}
%            a & b \\\hline
%            c & d
%        \end{tabular}
%    \end{center}
%\end{frame}

%==========================================================================


%==========================================================================


%==========================================================================

\begin{frame}[fragile]

  \vspace{10pt}

  \textbf{How to Get Your Fixes and Features into Kokkos}
  \newline
  \begin{itemize}
    \item Fork the Kokkos repo (\url{https://github.com/kokkos/kokkos})
    \item Make topic branch from \textit{develop} for your code
    \item Add tests for your code
    \item Create a pull request (PR) on the main project \textit{develop}
    \item Update the documentation (\url{https://github.com/kokkos/kokkos-core-wiki}) if your code changes the API
    \item Get in touch if you have any question (\url{https://kokkosteam.slack.com})
  \end{itemize}

\end{frame}

%==========================================================================

\end{document}
