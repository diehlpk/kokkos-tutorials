%==========================================================================

\begin{frame}[fragile]

  {\Huge Feature highlights}

  \vspace{10pt}

\end{frame}

%==========================================================================

% Examples

% note: always keep the [fragile] for your frames!

%\begin{frame}[fragile]{Example list}
%  \begin{itemize}
%      \item Item 1
%      \item Item 2 with some \texttt{code}
%      \begin{itemize}
%        \item Sub-item 2.1
%        \item Sub-item 2.2
%      \end{itemize}
%  \end{itemize}
%\end{frame}

%\begin{frame}[fragile]{Example code}
%    \begin{code}[keywords={std}]
%        #include <iostream>
%        
%        int main() {
%            std::cout << "hello world\n";
%        }
%    \end{code}
%\end{frame}

%\begin{frame}[fragile]{Example table}
%    \begin{center}
%        \begin{tabular}{l|l}
%            a & b \\\hline
%            c & d
%        \end{tabular}
%    \end{center}
%\end{frame}

%==========================================================================

\begin{frame}[fragile]\label{sec:new_features}

  {\Huge Kokkos::Graph features}

  \vspace{10pt}

\end{frame}

\begin{frame}[fragile]{Kokkos::Graph recap}
 \begin{itemize}
     \item describes asynchronous workloads organised as a direct acyclic graph (DAG)
     \item executed using \texttt{submit()}, possibly many times, observing dependencies
      \begin{code}[keywords={auto}]
auto graph = Kokkos::create_graph([&](auto root) {
    auto node_A = root.then_parallel_for("A", ...policy..., ...functor...);
    auto node_B = node_A.then_parallel_for("B", ...policy..., ...functor...);
    auto node_C = node_A.then_parallel_for("C", ...policy..., ...functor...);

    auto node_D = Kokkos::when_all(node_B, node_C).
                  then_parallel_for("D", ...policy..., ...functor...);
});

graph.instantiate();

graph.submit();
      \end{code}
 \end{itemize}
\end{frame}

\begin{frame}[fragile]{Kokkos::Graph new features}
 \begin{itemize}
  \item \href{https://github.com/kokkos/kokkos/pull/7629}{\texttt{then} node}: executes a callable on device
   \item Executed in the \texttt{ExecutionSpace} the graph is submitted to
     \begin{code}[keywords={auto}]
auto graph = Kokkos::create_graph([&](auto root) {
    auto node_A = root.then_parallel_for("A", ...policy..., ...functor...);
    auto node_B = node_A.then(...functor...);
});

graph.instantiate();

graph.submit();
     \end{code}
 \end{itemize}
\end{frame}

\begin{frame}[fragile]{Kokkos::Graph new features}
 \begin{itemize}
   \item \href{https://github.com/kokkos/kokkos/pull/7664}{interoperability}: create a \texttt{Kokkos::Graph} from a native Cuda/HIP/Sycl graph
   \item Does not yet allow linking Kokkos nodes to native nodes
     \begin{code}[keywords={auto}]
cudaGraph_t native_graph = nullptr;
KOKKOS_IMPL_CUDA_SAFE_CALL(cudaGraphCreate(&native_graph, 0));
auto graph_from_native =
  Kokkos::Experimental::create_graph_from_native(exec, native_graph);
auto root = Kokkos::Impl::GraphAccess::create_root_ref(graph_from_native);
root.then_parallel_for(1, Increment<view_t>{data});
...
graph_from_native.submit(exec);
     \end{code}
 \end{itemize}
\end{frame}

\begin{frame}[fragile]\label{sec:new_features}

  {\Huge Multi-GPU for HIP Backend}

  \vspace{10pt}

\end{frame}

\begin{frame}[fragile]{Multi-GPU for HIP Backend}
  \begin{itemize}
    \item Launch kernels on multiple devices from a single host process
    \item Available for ROCm 5.6 and later
    \item Requires knowledge of HIP API functions for creating and destroying streams
    \item Experimental, always looking for feedback from new users
    \item Newly documented (for all backends)  
      \begin{itemize} 
        \item[] \url{https://kokkos.org/kokkos-core-wiki/API/core/MultiGPUSupport.html}
      \end{itemize}
  \end{itemize}
\end{frame}

\begin{frame}[fragile]{Multi-GPU for HIP Backend}
  \begin{code}[keywords={auto}]
// Create streams on different devices
hipStream_t streams[2];
hipSetDevice(0); hipStreamCreate(&streams[0]);
hipSetDevice(1); hipStreamCreate(&streams[1]);
{
  // Creating execution spaces 
  Kokkos::HIP exec0(streams[0]), exec1(streams[1]);

  // Allocating views
  Kokkos::View<int*> v0(Kokkos::view_alloc("v0", exec0), N);
  Kokkos::View<int*> v1(Kokkos::view_alloc("v1", exec0), M);

  // Launch kernels (run concurrently)
  Kokkos::parallel_for(Kokkos::RangePolicy(exec0, 0, N), functor0);
  Kokkos::parallel_for(Kokkos::RangePolicy(exec1, 0, M), functor1);
}
// Destroy streams (after execution spaces are deleted)
hipStreamDestroy(streams[0]); hipStreamDestroy(streams[1]);
  \end{code}
\end{frame}

%==========================================================================
